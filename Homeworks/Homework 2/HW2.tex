\documentclass{article}
\usepackage{graphicx} % Required for inserting images
\usepackage[russian]{babel}

\title{Мини-конспект по теме: Теорема Пифагора}
\author{Я Сам}
\date{16 июля 2025 г.}

\begin{document}

\maketitle

\tableofcontents
\newpage

\section {Введение}
\paragraph {Теорема Пифагора — одна из важнейших теорем евклидовой геометрии. Она находит
применение в самых разных областях:}
\begin {itemize}
    \item геометрия и тригонометрия
    \item физика
    \item инженерные расчёты
    \item компьютерная графика
\end {itemize}

\section {Формулировка теоремы}
\paragraph {\textbf{Слова:} В прямоугольном треугольнике квадрат гипотенузы равен сумме квадратов
катетов.}

\paragraph {\begin{center} $c^2=a^2+b^2$ \end{center}\begin{flushright} (2) \end{flushright}}
\paragraph {Как видно из формулы 1, знание двух сторон позволяет найти третью.}
\section {Доказательство (набросок)}
\paragraph{Одно из доказательств основывается на площади квадрата, составленного из четырёх одинаковых прямоугольных треугольников и малого квадрата в центре. Раскладывая площадь двумя способами, получаем $c^2=a^2+ b^2$ }
\section {Примеры расчёта}
\subsection *{Пример 1}
\paragraph{\begin{center}
$a = 3$, $b = 4$\\
$c = \sqrt{a^2 + b^2} = \sqrt{9 + 16} = 5$
\end{center}}
\subsection *{Пример 2}
\begin{enumerate}
    \item Дано: $a = 5$, $b = 12$
    \item Решение:
\end{enumerate}
\paragraph{\begin{center}
$c = \sqrt{5^2 + 12^2} = \sqrt{25 + 144} = 13$\end{center}}
\section {Таблица значений}
\begin {centre}
\begin {tabular} {|c|c|c|}
\hline
Катет a&Катет b&Гипотенуза c\\
\hline 3&4&5\\
\hline 5&12&13\\
\hline 7&24&25\\
\hline
\end{tabular}
\end {centre}
\section {Иллюстрация}
\paragraph{Ниже пример изображения:\\
\begin {centre}
\includegraphics[width=0.5\textwidth]{triangle.png}
\end {centre}
}
\section {Заключение}
\paragraph{Теорема Пифагора — один из краеугольных камней геометрии, помогающий решать множество практических задач.}
\section {Ссылки и литература}
\begin{itemize}
\item \href{https://ru.wikipedia.org/wiki/%D0%A2%D0%B5%D0%BE%D1%80%D0%B5%D0%BC%D0%B0_%D0%9F%D0%B8%D1%84%D0%B0%D0%B3%D0%BE%D1%80%D0%B0}{Википедия: Теорема Пифагора}
\item Классические учебники геометрии
\end{itemize}
\end{document}